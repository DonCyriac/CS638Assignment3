\documentclass[a4paper, 12pt]{article}

\usepackage{escexam}

%\excludecomment{solution}

%\renewcommand*\ttdefault{cmvtt}

\begin{document}

\vspace*{14ex}

\makeheader{1}                              					% examination number (used to set theorem, lemma numbers)
           {March 15, 2020}      					         		% examination date or deadline
					 {40}											% total marks
					 {Homework Assignment 3}							% Minor Quiz 1, Major Quiz 2, End sem, etc
					
\begin{tabular}{cl}
1. & This question paper contains a total of 14 pages (14 sides of paper). Please verify.\\
2. & Write your name, roll number, department on \textbf{every side of every sheet} of this booklet\\
%3. & Write final answers \textbf{neatly with a pen} in the given boxes.\\
%4. & Do not give derivations/elaborate steps unless the question specifically asks you to provide these.
\end{tabular}


\begin{problem} {\textbf{Model-checker as a motion planner}} (20 points)

%In this assignment, we will use a model checker as a motion planner.
\bigskip
Consider a 2D workspace which
is divided into small rectangular blocks using a grid. The size of the workspace is $5 \times 5$. The lower left grid block has the ID $(0,0)$, and 
the upper right grid block has the ID $(4,4)$. The blocks $(2,0)$, $(3,0)$, $(1,2)$, $(3,2)$, $(1,4)$, and $(2,4)$ are covered with obstacles.
We have two robots whose initial locations are $(0,0)$ and $(4,4)$, respectively. 
The robots have four motion primitives: L, R, U, D that can take the robot from its current block location to the left, right, upper and lower block respectively.  
The robots have to move to the blocks $(4,4)$ and $(0,0)$, respectively. Moreover, The second robot should reach its destination $(0,0)$ strictly after the first robot  reaches its destination.
Capture the behavior of the robots as a transition system and the requirement stated above as an LTL formula. 
Then through model checking, synthesize a trajectory for the robots. Use NuSMV model checker.

\medskip
\noindent
Submit the following:
\begin{itemize}
\item NuSMV model and specification.
\item A snapshot of the terminal showing the execution of the model-checker.
\item Provide a visual representation of the trajectories synthesized by NuSMV.
\end{itemize}

\medskip
\noindent
NuSMV Wbdpage:
\url{http://nusmv.fbk.eu}

\newpage
\ \\
\begin{minipage}{1\textwidth}
		\rectangle{\linewidth}{24cm}
		% \ruledrectangle{7}
\end{minipage}
\newpage
\ \\
\begin{minipage}{1\textwidth}
		\rectangle{\linewidth}{24cm}
		% \ruledrectangle{7}
\end{minipage}
\newpage
\ \\
\begin{minipage}{1\textwidth}
		\rectangle{\linewidth}{24cm}
		% \ruledrectangle{7}
\end{minipage}
\newpage
\ \\
\begin{minipage}{1\textwidth}
		\rectangle{\linewidth}{24cm}
		% \ruledrectangle{7}
\end{minipage}
\newpage
\ \\
\begin{minipage}{1\textwidth}
		\rectangle{\linewidth}{24cm}
		% \ruledrectangle{7}
\end{minipage}
\newpage
\ \\
\begin{minipage}{1\textwidth}
		\rectangle{\linewidth}{24cm}
		% \ruledrectangle{7}
\end{minipage}
\end{problem}



\newpage
\begin{problem} {\textbf{Reactive motion planning}} (20 points)

%In this assignment, we will learn how to synthesize a reactive motion plan for a robot.
\bigskip
Consider an office with 7 rooms for work and one kitchen room. A robot has been entrusted with 
the responsibility of collecting used coffee mugs from the rooms and bring them to the kitchen.
The robot can carry only one cup at a time. It keeps on visiting the rooms and if there is an empty cup in a room,
it brings it to the kitchen. If it does not find a cup, it visits another room.

Capture the requirements stated above in the form of an LTL formula. 
Construct the layout of the office space based on the layout shown in Figure 2 in [KFP09],
where region 1 is the kitchen and region 2-8 are the office rooms.
Then with the help of LTLMoP tool, synthesize a reactive controller for the robot and simulate its behavior.

[KFP09] H. Kress-Gazit, G. E. Fainekos, and G. J. Pappas. Temporal-logic-based reactive mission and motion planning. IEEE Transactions on Robotics, 25(6):1370-1381, 2009. 

\bigskip
\noindent
Submit the following:
\begin{itemize}
\item Specification using LTL symbols.
\item Specification in LTLMoP syntax.
\item Synthesized Controller.
\item Snapshot of the trajectories.
\end{itemize}

\bigskip
\noindent
LTLMoP Webpage:
\url{https://ltlmop.github.io}

\newpage
\ \\
\begin{minipage}{1\textwidth}
		\rectangle{\linewidth}{24cm}
		% \ruledrectangle{7}
\end{minipage}
\newpage
\ \\
\begin{minipage}{1\textwidth}
		\rectangle{\linewidth}{24cm}
		% \ruledrectangle{7}
\end{minipage}
\newpage
\ \\
\begin{minipage}{1\textwidth}
		\rectangle{\linewidth}{24cm}
		% \ruledrectangle{7}
\end{minipage}
\newpage
\ \\
\begin{minipage}{1\textwidth}
		\rectangle{\linewidth}{24cm}
		% \ruledrectangle{7}
\end{minipage}
\newpage
\ \\
\begin{minipage}{1\textwidth}
		\rectangle{\linewidth}{24cm}
		% \ruledrectangle{7}
\end{minipage}
\newpage
\ \\
\begin{minipage}{1\textwidth}
		\rectangle{\linewidth}{24cm}
		% \ruledrectangle{7}
\end{minipage}
\end{problem}


% %\toptitlebar
% \begin{center}
% 	BLANK SPACE: Any answers written here will be left ungraded.\\ No exceptions.\\ You may use this space for rough work.
% \end{center}
% \begin{figure}[h!]
% \includegraphics[width=\columnwidth]{watermark.png}%
% \end{figure}

\end{document}